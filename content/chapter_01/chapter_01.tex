%!TEX root = ../../thesis_master.tex

%%%%%%%%%%
\chapter{Introduction}
\label{chap:introduction}
%%%%%%%%%%

%%%%%%%%%%
\section{Motivation and Background}
\label{sec:motivation-brackground}
%%%%%%%%%%

During the last decade, the use of \emph{Unmanned Aerial Vehicles} (UAVs) have spread among very different applications, since aerial robots can be very helpful to improve the way some tasks are already achieved by terrestrial platforms. For example, object manipulation and transportation, environment mapping or surveillance. At the Institute of Automation Engineering\footnote{Technische Universität Dresden. Institut für Automatisierungstechnik. Dresden, Germany} of the Technical University of Dresden, a new robotic platform called Flypulator is being developed in cooperation with the Institute of Solid Mechanics\footnote{Technische Universität Dresden. Institut für Festkörpermechanik. Dresden, Germany} to investigate the use of flying robots in aerial manipulation.

\nomenclature[ba]{UAV}{Unmanned Aerial Vehicle}
\nomenclature[ba]{TUD}{Technische Universität Dresden}

When dealing with manipulation of objects, it is desired that the aerial robot adopts a certain pose with respect to the target before the proper manipulation process starts. The present work develops a \emph{Visual Servoing} (VS) control system that helps an aerial robot to acquire the desired pose by means of image data.

\nomenclature[ba]{VS}{Visual Servoing}

A monocular monochrome camera and an IMU are the only available on board sensor. For the controller proposed, the feedback is directly computed from image features rather than estimating the robot's pose and using the pose errors as control input.

\nomenclature[ba]{IMU}{Inertial Measurement Unit}

Vision results to be a passive (in contrast to GPS) and cheap sensor (in contrast to LIDAR). Visual odometry performs well navigating in GPS denied environments like indoors, but its not appropriated to regulate the relative navigation of the vehicle with respect to a target.  

\nomenclature[ba]{GPS}{Global Positioning System}
\nomenclature[ba]{LIDAR}{Light Detection and Ranging}

In order to integrate the visual servoing algorithm into the future modular robot system, the algorithm has been designed and tested on a conventional quadrotor. The aerial robot is implemented within the ROS framework, where the visual servo controller developed for this thesis is also integrated. Instead of using real hardware the complete system is simulated using the Gazebo robotics simulator.

\nomenclature[ba]{ROS}{Robot Operative System}

%%%%%%%%%%
\section{Aims and Objectives}
\label{sec:aims-objectives}
%%%%%%%%%%

The aim of this work is to implement and test an \emph{Image Based Visual Servoing} (IBVS) control algorithm for a quadrotor, which could be later used by the Flypulator project. This includes the review of the state of the art with regard to \emph{Visual Servoing}, the design of a solution and a prototypical implementation with in the ROS framework and simulation with Gazebo of a test case.

\nomenclature[ba]{ROS}{Structure Analysis}

The present thesis documents comprehensively the theoretical background, implementation details and results of the conducted work through the following structure. In Chapter \ref{chap:theory-state-art} the theoretical background and state of the art of Visual Servoing is presented. Chapter \ref{chap:srs-sa} contains the \emph{Software Requirements Specification} of the system developed as well as its decomposition by \emph{Structured Analysis} method. Chapter \ref{chap:system-design} describes the solution designed and the algorithms to be tested. Chapter \ref{chap:implementation} deals with the system implementation, which is tested and validated in Chapter \ref{chap:system-validation} testing and validation. Finally, Chapter \ref{chap:conclusions-future-work} presents the conclusions of the project and suggests future improvements and research paths.